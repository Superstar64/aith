\documentclass {article}
\usepackage{mathtools}
\usepackage{mathabx}
\usepackage{hyperref}
\allowdisplaybreaks
\begin{document}

As of now, this document is mostly for myself. Its missing a lot of prose and explantions. \\

\tableofcontents


\section{Main Language}
All valid terms have a type who's kind is of the form $ * $. \\
\\
All valid types have a kind who's sort is of the form $ \boxvoid^\mu_{\mu'} $.
where $ \mu $ is an $ \text{Orderability} $ ($\text{Invariant} $ or $ \text{Subtypable}$)
and $ \mu' $ is a $ \text{Transparency}$ ($\text{Transparent}$ or $\text{Opaque}$). \\
\\
Variables in the type environment are annotated with their multiplicity. ie $ x :^1 ... $. \\
\\
Function literals and function literal types can not appear at runtime. They must be bound by a global "function" binding.
This binding converts a function literal into an extern and a function literal type into a function pointer type. \\
\\
Terms with a type $ \sigma^\pi $ are runtime terms. \\
\\
Overlines are used to mean 0 or more. \\
\\
After $ \beta $ reduction, terms not of type $ \sigma^\pi $ must either be a free variable or correspond to these $ \beta $ normal forms. \\
\begin{gather*}
\begin{array}{c | c}
\text{Type} & \text{Term} \\
\hline
\sigma \to \tau & \lambda x. e \\
!\sigma & !e \\
\forall \alpha : \kappa \succeq \{ \overline \pi \}. \sigma & \Lambda \alpha. e  \\
\forall \beta : \mu. \sigma & \Lambda \beta. e \\
\sigma \xrightarrow{\text{text}} \tau & \text{function}(x). e \\
\end{array}
\\
\end{gather*}

\subsection{Object Categories}

\begin{align*}
e \tag{Terms} \\
x \tag{Variables} \\
n \tag{Numbers} \\
\sigma, \tau, \pi, \kappa, s, \rho, \mu \tag{Types, Kinds, and Sorts} \\
\alpha, \beta \tag{Type and Kind Variables} \\
\Gamma \tag{Type Environment} \\
\Phi \tag{Misc Environment} \\
\text{sym} \tag{Symbol}
\end{align*}


\subsection{Judgment Forms}
The term rules are not syntax directed.
The type of a term must be a valid type ($ \Phi | \Gamma \vdash e : \sigma $, requires $ \Phi \vdash \sigma : \kappa $).
The type checking can be checked using synthesis mode.
\begin{align*}
\Phi \mid \Gamma \vdash \, & e : \sigma \tag{\text{Term Validation}} \\
\Phi \vdash \, & \sigma : \kappa \tag{\text{Type Validation}} \\
\Phi \vdash \, & \sigma \succeq \tau \tag{Subtyping}
\end{align*}

\subsection{Typing Environments}
\begin{align*}
\Gamma \Coloneqq & \\
& \Gamma, x :^\pi \sigma \\
& \emptyset \\
\Phi \Coloneqq & \\
& \Phi, \alpha : \kappa \succeq \{ \overline \pi \} \\
& \emptyset \\
\end{align*}

\subsection{Meta}
\begin{align*}
\sigma, \tau, \pi, \kappa, s, \rho, \mu \Coloneqq & \\
& * \tag{Type} \\
& \boxvoid^\mu_{\mu'} \tag{Kind} \\
& \text{Invariant} \\
& \text{Subtypable} \\
& \text{Transparent} \\
& \text{Opaque} \\
& 1 \tag{Linear} \\
& \omega \tag{Unrestricted} \\
& \text{Multiplicity} 
\end{align*}

\begin{gather*}
\frac
{}
{\Phi \vdash * : \boxvoid^\text{Invariant}_{\text{Transparent}}} \\
\\
\frac
{}
{\Phi \vdash 1 : \text{Multiplicity}} \\
\\
\frac
{}
{\Phi \vdash \omega : \text{Multiplicity}} \\
\\
\frac
{}
{\Phi \vdash \text{Multiplicity} : \boxvoid^\text{Subtypable}_\text{Transparent}}
\end{gather*}

\subsubsection{Meta (unused)}
\begin{align*}
\sigma, \tau, \pi, \kappa, s, \rho, \mu \Coloneqq & \\
& \Delta \tag{Sort} \\
& \text{Orderability} \\
& \text{Transparency}
\end{align*}

\begin{gather*}
\frac
{\Phi \vdash \mu : \text{Substitutability} \quad \Phi \vdash \mu' : \text{Orderability} }
{\Phi \vdash \boxvoid^\mu_{\mu'} : \Delta} \\
\\
\frac
{}
{\Phi \vdash \text{Invariant} : \text{Orderability}} \\
\\
\frac
{}
{\Phi \vdash \text{Subtypable} : \text{Orderability}} \\
\\
\frac
{}
{\Phi \vdash \text{Transparent} : \text{Transparency}} \\
\\
\frac
{}
{\Phi \vdash \text{Opaque} : \text{Transparency}}
\end{gather*}
    

\subsubsection{Variables}
\begin{align*}
e \Coloneqq & \\
& x \tag{Variable} \\
& \dots \\
\sigma, \tau, \pi, \kappa, s, \rho \Coloneqq & \\
& \alpha \tag{Type Variable} \\
& \dots
\end{align*}

\begin{gather*}
% todo use reference count analysis, ala making uniqueness typing less unique, rather then substructural rules
\Phi | x :^\pi \sigma \vdash x : \sigma \\
\\
\frac
{\Phi | \Gamma, \Gamma' \vdash e : \sigma}
{\Phi | \Gamma', \Gamma \vdash e : \sigma} \\
\\
\frac
{\Phi | \Gamma \vdash e : \sigma}
{\Phi | \Gamma, x :^\omega \tau \vdash e : \sigma } \\
\\
\frac
{\Phi | \Gamma, x :^\omega \tau, x :^\omega \tau \vdash e : \sigma}
{\Phi | \Gamma, x :^\omega \tau \vdash e : \sigma } \\
\end{gather*}

\begin{gather*}
\frac
{(\alpha : \kappa \succeq \{ \overline \pi \} ) \in \Phi}
{\Phi \vdash \alpha : \kappa}
\end{gather*}

\subsubsection{Macro Linear Lambda Calculus}
\begin{align*}
e \Coloneqq & \\
& \lambda^\uparrow x. e \tag{Macro Lambda}\\
& e(^\uparrow e') \tag{Macro Application}\\
& \dots \\
\sigma, \tau, \pi, \kappa, s, \rho \Coloneqq & \\
& \sigma \xrightarrow{\uparrow}_\pi \tau \tag{Macro} \\
& \dots \\
\dots
\end{align*}

\begin{gather*}
\frac
{\Phi | \Gamma, x :^\pi \sigma \vdash e : \tau}
{\Phi | \Gamma \vdash \lambda^\uparrow x. e : \sigma \xrightarrow{\uparrow}_\pi \tau } \\
\\
\frac
{\Phi | \Gamma \vdash e_1 : \sigma \xrightarrow{\uparrow}_\pi \tau \quad \Phi | \Gamma' \vdash e_2 : \sigma \quad \Gamma \succeq \pi}
{\Phi | \Gamma, \Gamma' \vdash e_1(^\uparrow e_2) : \tau}
\end{gather*}

\begin{gather*}
\frac
{\Phi \vdash \sigma : * \quad \Phi \vdash \tau : * \quad \Phi \vdash \pi : \text{Multiplicity}}
{\Phi \vdash \sigma \xrightarrow{\uparrow}_\pi \tau  : * }
\end{gather*}

\subsubsection{System-F}
\begin{align*}
e \Coloneqq & \\
& \Lambda \alpha. e \tag{Type Lambda} \\
& e (\sigma) \tag{Type Application} \\
& \dots \\
\sigma, \tau, \pi, \kappa, s, \rho, \mu \Coloneqq & \\
& \forall \alpha : \kappa \succeq \{ \overline \pi \}. \sigma \tag{Type Poly}\\ 
& \dots
\end{align*}

\begin{gather*}
\frac
{\Phi, \alpha : \kappa  \succeq \{ \overline \pi \} | \Gamma \vdash e : \sigma}
{\Phi | \Gamma \vdash \Lambda \alpha. e : \forall \alpha : \kappa \succeq \{ \overline \pi \} . \sigma } \\
\\
\frac
{\Phi | \Gamma \vdash e : \forall \alpha \succeq \{ \overline \pi \} : \kappa. \sigma \quad \Phi \vdash \tau : \kappa\quad \overline { \Phi \vdash \tau \succeq \pi }}
{\Phi | \Gamma \vdash e (\tau) : \sigma [\tau/\alpha] }
\end{gather*}

\begin{gather*}
\frac
{\splitfrac{\Phi, \alpha : \kappa \succeq \{ \overline \pi \}\vdash \sigma : * \quad \overline { \Phi \vdash \pi : \kappa}}
{\splitfrac{\Phi \vdash \kappa : \boxvoid^\mu_{\mu'} \quad \text{when $ \pi $ is empty}}
{\Phi \vdash \kappa : \boxvoid^\text{Subtypable}_{\mu'} \quad \text{when $ \pi $ is not empty} }}}
{\Phi \vdash \forall \alpha : \kappa + \{ \overline c \}. \sigma : * } \\
\end{gather*}

\subsection{Runtime}

\begin{align*}
\sigma, \tau, \pi, \kappa, s, \rho, \mu \Coloneqq & \\
& +^\rho_\pi \tag{Pretype}\\
& \text{Representation} \tag {Representation}\\
& \dots
\end{align*}

\begin{gather*}
\\
\frac
{\Phi \vdash \rho : \text{Representation} \quad \Phi \vdash \pi : \text{Multiplicity}}
{\Phi \vdash +^\rho_\pi : \boxvoid^\text{Invariant}_\text{Transparent}} \\
\\
\frac
{}
{\Phi \vdash \text{Representation} : \boxvoid^\text{Invariant}_\text{Opaque}}
\end{gather*}

\subsubsection{Single Effect Regions}
\begin{align*}
e \Coloneqq & \\
& \text{letRGN} \, (\Lambda \alpha. e) \tag{Create Region} \\
& \text{let}^\downarrow \, x = e; e' \tag{Runtime Let} \\
& \dots \\
\sigma, \tau, \pi, \kappa, s, \rho \Coloneqq & \\
& \sigma^\pi \tag{Region Effect} \\
& \text{IO} \tag{IO Region} \\
& \text{Region} \tag{Region} \\
& \dots
\end{align*}

\begin{gather*}
\frac
{\Phi , \alpha : \text{Region} \, \succeq \{ \pi \} | \Gamma \vdash e : \sigma^\alpha \quad \alpha \notin \text{Free}(\sigma, \pi)}
{\Phi | \Gamma \vdash \text{letRGN} \, (\Lambda \alpha. e) : \sigma^\pi} \\
\\
\frac
{\Phi | \Gamma \vdash e : \tau^\pi \quad \Phi | \Gamma', x :^{\pi'} \forall \alpha. \tau^\alpha \vdash e' : \sigma^\pi \quad \Phi \vdash \tau : +^{\rho}_{\pi'}}
{\Phi | \Gamma, \Gamma' \vdash \text{let}^\downarrow \, x = e; e' : \sigma^\pi}
\end{gather*}

\begin{gather*}
\frac
{\Phi \vdash \pi : \text{Region} \quad \Phi \vdash \sigma : +^\rho_\mu}
{\Phi \vdash \sigma^\pi : *} \\
\\
\frac
{}
{\Phi \vdash \text{IO} : \text{Region}}
\end{gather*}

\begin{gather*}
\frac
{}
{\Phi \vdash \text{Region} : \boxvoid^\text{Subtypable}_\text{Transparent}} \\
\end{gather*}

\subsubsection{Boxed}
\begin{align*}
e \Coloneqq & \\
& \text{borrow} (e) \text{as} \, \Lambda \alpha. (x) \{ e' \} \\
\sigma, \tau, \pi \Coloneqq & \\
& \text{unique} \, \sigma \\
& \sigma \, @ \, \pi \\
\kappa, s, \rho \Coloneqq & \\
& - \tag{Boxed} \\
& \text{Pointer} \tag{Pointer Representation} \\
& \dots \\
\end{align*}

\begin{gather*}
\frac
{\Phi | \Gamma \vdash e : (\text{unique} \, \tau)^\pi \quad \Phi,\alpha : \text{Region} \, \succeq \{ \pi \} \, | \, \Gamma', x :^\omega \forall \alpha'. (\tau @ \alpha)^{\alpha'} \vdash e' : \sigma^\alpha \quad \alpha \notin \text{Free}(\sigma, \pi)}
{\Phi | \Gamma, \Gamma' \vdash \text{borrow} (e) \text{as} \, \Lambda \alpha. (x) \{ e' \} : (\sigma, \text{unique} \, \tau)^\pi }
\end{gather*}

\begin{gather*}
\frac
{\Phi \vdash \sigma : -}
{\Phi \vdash \text{unique} \, \sigma : +^\text{Pointer}_1} \\
\\
\frac
{\Phi \vdash \sigma : - \quad \pi : \text{Region}}
{\Phi \vdash \sigma \, @ \, \pi : +^\text{Pointer}_\omega} 
\end{gather*}

\begin{gather*}
\frac
{}
{\Phi \vdash - : \boxvoid^\text{Invariant}_\text{Transparent}} \\
\\
\frac
{}
{\Phi \vdash \text{Pointer} : \text{Representation}} \\
\end{gather*}

\subsubsection{Pointers}
\begin{align*}
e \Coloneqq & \\
& *e \tag{Read Pointer} \\
& *e = e' \tag{Write Pointer} \\
& \& * \tag{Array to Pointer} \\
& \& e [e'] \tag{Array Increment} \\
& \dots \\
\sigma, \tau, \pi, \kappa, s, \rho, \mu \Coloneqq & \\
& \sigma* \tag{Pointer} \\
& \sigma[] \tag{Array Pointer} \\
\end{align*}

\begin{gather*}
\frac
{\Phi | \Gamma \vdash e : (\sigma* \, @ \pi')^\pi \quad \Phi \vdash \text{Copy} (\sigma) \quad \pi \succeq \pi'}
{\Phi | \Gamma \vdash * e  : \sigma^\pi} \\
\\
\frac
{\Phi | \Gamma \vdash e : (\sigma* \, @ \pi')^\pi \quad \Phi | \Gamma \vdash e' : \sigma^\pi  \quad \Phi \vdash \text{Copy} (\sigma) \quad \pi \succeq \pi'}
{\Phi | \Gamma \vdash *e  = e' : ()^\pi } \\
\\
\frac
{\Phi | \Gamma \vdash e : (\sigma[] @ \pi') ^ \pi}
{\Phi | \Gamma \vdash \& * e : (\sigma * @ \pi') ^ \pi}\\
\\
\frac
{\Phi | \Gamma \vdash e : (\sigma [] @ \pi')^\pi \quad \Phi |
 \Gamma' \vdash e' : (\text{unsigned} \, \text{integer}(\text{native}))^\pi}
{\Phi | \Gamma, \Gamma' \vdash \&e [e'] : (\sigma [] @  \pi')^\pi}
\end{gather*}

\begin{gather*}
\frac
{\Phi \vdash \sigma : +^{\rho}_\sigma}
{\Phi \vdash \sigma* : -} \\
\\
\frac
{\Phi \vdash \sigma : +^{\rho}_\sigma}
{\Phi \vdash \sigma [] : -} \\
\end{gather*}

\subsubsection{Functions}

\begin{align*}
e \Coloneqq & \\
& \text{extern} \, \text{sym} \tag{Extern Function} \\
& e ^\downarrow (e') \tag{Function Pointer Application} \\
& \text{function} (x). e \tag{Function Literal} \\
& \dots \\
\sigma, \tau, \pi \Coloneqq & \\
& \tau \xrightarrow{\downarrow}_\pi \sigma \tag{Function Pointer}\\
& \tau \xrightarrow{\text{text}}_\pi \sigma  \tag{Function Literal Type} \\
& \dots
\end{align*}

\begin{gather*}
\frac
{}
{\Phi | \Gamma \vdash \text{extern} \, \text{sym} : (\tau \xrightarrow{\downarrow}_{\pi'} \sigma)^\pi} \\
\\
\frac
{\Phi | \Gamma \vdash e : (\sigma \xrightarrow{\downarrow}_{\pi'} \tau)^\pi \quad \Phi | \Gamma' \vdash e' : \sigma^\pi \quad \pi \succeq \pi'}
{\Phi | \Gamma, \Gamma' \vdash e^\downarrow (e') : \tau^\pi} \\
\\
\frac
{\Phi | \Gamma, x :^1 \forall \alpha. \sigma^\alpha \vdash e : \tau^\pi}
{\Phi | \Gamma \vdash \text{function}(x). e : \sigma \xrightarrow{\text{text}}_\pi \tau} \\
\end{gather*}

\begin{gather*}
\frac
{\Phi \vdash \sigma : +^{\rho}_\mu \quad \Phi \vdash \tau : +^{\rho'}_{\mu'} \quad \Phi \vdash \pi : \text{Region}}
{\Phi \vdash \sigma \xrightarrow{\downarrow}_\pi \tau : +^\text{Pointer}_\omega} \\
\\
\frac
{\Phi \vdash \sigma : +^{\rho}_\mu \quad \Phi \vdash \tau : +^{\rho'}_{\mu'} \quad \Phi \vdash \pi : \text{Region}}
{\Phi \vdash \sigma \xrightarrow{\text{text}}_\pi \tau : *} \\
\end{gather*}

\subsubsection{Tuples}
\begin{align*}
e \Coloneqq & \\
& (\overline e,) \tag{Tuple Introduction} \\
& \text{let}^\downarrow (\overline x) = e; e' \tag{Tuple Elimination} \\
& \dots \\
\sigma, \tau, \pi, \kappa, s, \rho, \mu \Coloneqq & \\
& (\overline \sigma,)^\mu \tag{Tuple} \\
& \text{Struct} \{ \overline \rho \} \tag{Struct Representation} \\
& \dots
\end{align*}

\begin{gather*}
\frac
{\Phi | \overline { \Gamma \vdash e : \sigma^\pi }}
{\Phi | \overline \Gamma  \vdash (\overline e,) : ((\overline \sigma,)^\tau)^\pi } \\
\\
\frac
{\Phi | \Gamma \vdash e : ((\overline \tau,)^\mu)^\pi \quad \Phi | \Gamma , \overline {x :^1 \forall \alpha. \tau^\alpha} \vdash e : \sigma^\pi}
{\Phi | \Gamma, \Gamma' \vdash \text{let}^\downarrow (\overline x) = e; e' : \sigma^\pi} \\
\\
\frac
{\Phi \vdash \overline {\sigma : +^{\kappa}_{\mu'}} \quad \Phi \vdash \mu : \text{Multiplicity} \quad \Phi \vdash \overline {\mu'} \succeq \mu }
{\Phi \vdash (\overline \sigma,)^\mu : +^{\text{Struct} \{ \overline \kappa\}}_\mu} \\
\\
\frac
{\Phi \vdash \overline {\rho : \text{Representation}}}
{\Phi \vdash \text{Struct} \{ \overline \rho \} : \text{Representation} } \\
\end{gather*}

\subsubsection{Choices}
\begin{align*}
\sigma, \tau, \pi, \kappa, s, \rho, \mu \Coloneqq & \\
& \text{Union} \{ \overline \rho \} \tag{Union Representation} \\
& \dots
\end{align*}

\begin{gather*}
\frac
{\Phi \vdash \overline {\rho : \text{Representation}}}
{\Phi \vdash \text{Union} \{ \overline \rho \} : \text{Representation} } \\
\end{gather*}

\subsubsection{Integer Arithmatic}
\begin{align*}
e \Coloneqq & \\
& n \tag{Numeric Literal} \\
& e + e' \tag{Addition} \\
& e - e' \tag{Subtraction} \\
& e * e' \tag{Multiplication} \\
& e / e' \tag{Division} \\
& e = e' \tag {Equality} \\
& e \neq e' \tag {Inequality} \\
& e < e' \tag {Less Then} \\
& e \leq e' \tag {Less Then Equal} \\
& e > e' \tag {Greater Then} \\
& e \geq e' \tag {Greater Then Equal} \\
& \dots \\
\sigma, \tau, \pi \Coloneqq & \\
& \rho \, \text{integer}(\rho) \tag{Number} \\
& \dots \\
\sigma, \tau, \pi, \kappa, s, \rho, \mu \Coloneqq & \\
& \text{Word} (\rho) \tag{Word Representation} \\
& 8 \tag{Byte Size} \\
& 16 \tag{Short Size} \\
& 32 \tag{Int Size} \\
& 64 \tag{Long Size} \\
& \text{native} \tag{Native Size} \\
& \text{signed} \tag{Signed} \\
& \text{unsigned} \tag{Unsigned} \\
& \text{Size} \tag{Size Sort}\\
& \text{Signedness} \tag{Signedness Sort}\\
& \dots \\
\end{align*}

\begin{gather*}
\frac
{}
{\Phi | \Gamma \vdash n : (\rho \, \text{integer}(\rho'))^\pi} \\
\\
\frac
{\Phi | \Gamma \vdash e : (\rho \, \text{integer}(\rho'))^\pi \quad \Phi | \Gamma, \Gamma' \vdash e' : (\rho \, \text{integer}(\rho'))^\pi}
{\Phi | \Gamma, \Gamma' \vdash e + e' : (\rho \text{integer}(\rho'))^\pi} \\
\dots \\
\\
\frac
{\Phi | \Gamma \vdash e : (\rho \, \text{integer}(\rho'))^\pi \quad \Phi | \Gamma, \Gamma' \vdash e' : (\rho \, \text{integer}(\rho'))^\pi}
{\Phi | \Gamma, \Gamma' \vdash e < e' : \text{Boolean}^\pi} \\
\dots
\end{gather*}

\begin{gather*}
\frac
{\Phi \vdash \rho : \text{Signedness} \quad \Phi \vdash \rho' : \text{Size}}
{\Phi \vdash \rho \, \text{integer}(\rho') : +^ {\text{Word} (\rho')}_\omega } \\
\end{gather*}

\begin{gather*}
\frac
{\Phi \vdash \rho : \text{Size}}
{\Phi \vdash \text{Word} (\rho) : \text{Representation}} \\
\\
\frac
{}
{\Phi \vdash 8 : \text{Size}} \\
\\
\frac
{}
{\Phi \vdash 16 : \text{Size}} \\
\\
\frac
{}
{\Phi \vdash 32 : \text{Size}} \\
\\
\frac
{}
{\Phi \vdash 64 : \text{Size}} \\
\\
\frac
{}
{\Phi \vdash \text{native} : \text{Size}} \\
\\
\frac
{}
{\Phi \vdash \text{signed} : \text{Signedness}} \\
\\
\frac
{}
{\Phi \vdash \text{unsigned} : \text{Signedness}} \\
\\
\frac
{}
{\Phi \vdash \text{Size} : \boxvoid^\text{Invariant}_\text{Opaque}} \\
\\
\frac
{}
{\Phi \vdash \text{Signedness} : \boxvoid^\text{Invariant}_\text{Opaque}}
\end{gather*}

\subsubsection{Boolean Logic}
\begin{align*}
e \Coloneqq & \\
& \text{true} \\
& \text{false} \\
& \text{if} (e) \{ e' \} \text{else} \{ e'' \} \\
\sigma, \tau, \pi, \kappa, s, \rho, \mu \Coloneqq & \\
& \text{Boolean}
\end{align*}

\begin{gather*}
\frac
{}
{ \Phi | \Gamma \vdash \text{true} : \text{Boolean}^\pi} \\
\\
\frac
{}
{ \Phi | \Gamma \vdash \text{false} : \text{Boolean}^\pi} \\
\\
\frac
{\Phi | \Gamma \vdash e : \text{Boolean}^\pi \quad \Phi | \Gamma' \vdash e' : \sigma^\pi \quad \Phi | \Gamma' \vdash e'' : \sigma^\pi }
{\Phi | \Gamma, \Gamma' \vdash \text{if} (e) \{ e' \} \text{else} \{ e'' \} : \sigma^\pi }
\end{gather*}

\begin{gather*}
\frac
{}
{ \Phi \vdash \text{Boolean} : +^{\text{Word}(8)}_\omega}
\end{gather*}

\subsubsection{Loops}
\begin{align*}
e \Coloneqq & \\
& \text{continue} \, e \tag{continue} \\
& \text{break} \, e \tag{break} \\
& \text{loop} (\text{let} \, x = e) \{ e' \} \tag{loop} \\
\sigma, \tau, \pi, \kappa, s, \rho, \mu \Coloneqq & \\
& \text{Step} \, \sigma \, \tau \tag{Loop Instruction}
\end{align*}

\begin{gather*}
\frac
{\Phi | \Gamma \vdash e : \sigma^\pi }
{\Phi | \Gamma \vdash \text{continue} \, e : (\text{Step} \, \tau \, \sigma)^\pi } \\
\\
\frac
{\Phi | \Gamma \vdash e : \tau^\pi }
{\Phi | \Gamma \vdash \text{break} \, e : (\text{Step} \, \tau \, \sigma)^\pi } \\
\\
\frac
{\Phi | \Gamma \vdash e : \sigma^\pi \quad \Phi | \Gamma', x :^1 \forall \alpha. \sigma^\alpha \vdash e' : (\text{Step} \, \tau \, \sigma) ^\pi \quad \Gamma' \succeq \omega }
{\Phi | \Gamma, \Gamma' \vdash \text{loop} (\text{let} \, x = e) \{ e' \} : \tau^\pi } \\
\\
\frac
{\Phi \vdash \sigma : +^{\rho}_{\pi} \quad \Phi \vdash \tau : +^{\mu}_{\pi'} }
{\Phi \vdash \text{Step} \, \sigma \, \tau : +^{\text{Struct} (\text{Word} (8), \text{Union} (\rho, \mu) ) }_1 }
\end{gather*}

\subsection{Subtyping}
\begin{gather*}
\frac
{\Phi \vdash \sigma : \kappa \quad \Phi \vdash \kappa : \boxvoid^\text{Subtypable}_{\mu'}}
{\Phi \vdash \sigma \succeq \sigma} \\
\\
\frac
{\Phi \vdash \sigma \succeq \pi \quad \Phi \vdash \pi \succeq \tau}
{\Phi \vdash \sigma \succeq \tau} \\
\\
\frac
{(\alpha : \kappa \succeq  \{ \tau, \tau', \dots \sigma \}) \in \Phi }
{\Phi \vdash \alpha \succeq \sigma} \\
\\
\frac
{}
{\Phi \vdash \omega \succeq 1}
\end{gather*}

\section{Simple Language}
The is the internal intermediate language used before C code generation.
Like terms in the main language, this is typechecked in checking mode.
\begin{align*}
e \Coloneqq & \\
& x \\
& \text{let} \, x : \sigma = e; e' \\
& \text{extern} \, \text{sym} \, \sigma \to \tau \\
& e(e' : \tau) \\
& (e,e') \\
& \text{let} (x: \sigma, x' : \sigma') = e; e' \\
& * e \\
& * e = (e : \sigma) \\
& n \\
& e +^z e' \\
& e <^{sz} e' \\
& \text{true} \\
& \text{false} \\
& \text{if} (e) {e'} \text{else} {e''} \\
& \&e[e']^\sigma \\
& \text{continue} \, e \\
& \text{break} \, e \\
\sigma \Coloneqq & \\
& \text{void}* \\ 
& (\sigma, \sigma', \dots) \\
& \text{union} (\sigma, \sigma', \dots) \\
& \text{word} \, s \\
s \Coloneqq & \, \text{char} \, | \, \text{short} \, | \, \text{int} \, | \, \text{long} | \, \text{size} \\
z \Coloneqq & \, \text{signed} \, | \, \text{unsigned}
\end{align*}

\begin{gather*}
\frac
{(x : \sigma) \in \Gamma}
{\Gamma \vdash x : \sigma} \\
\\
\frac
{\Gamma \vdash x : \sigma \quad \Gamma, x : \sigma \vdash e' : \tau }
{\Gamma \vdash \text{let} \, x : \sigma = e; e' : \tau} \\
\\
\frac
{}
{\Gamma \vdash \text{extern} \, \text{sym} \, \sigma \to \tau : \text{void} *} \\
\\
\frac
{\Gamma \vdash e : \text{void} * \quad \Gamma \vdash e' : \tau}
{\Gamma \vdash e(e' : \tau) : \sigma} \\
\\
\frac
{\Gamma \vdash e : \sigma \quad \Gamma \vdash e' : \sigma'}
{\Gamma \vdash (e, e') : (\sigma, \sigma') } \\
\\
\frac
{\Gamma \vdash e : (\sigma, \sigma') \quad \Gamma, x : \sigma, x' : \sigma' \vdash e' : \tau}
{\Gamma \vdash \text{let} (x: \sigma, x' : \sigma') = e; e' : \tau} \\
\\
\frac
{\Gamma \vdash e : \text{void} *}
{\Gamma \vdash * e : \sigma} \\
\\
\frac {\Gamma \vdash e : \text{void} * \quad \Gamma \vdash e' : \sigma}
{\Gamma \vdash *e = (e' : \sigma) : ()} \\
\\
\frac
{}
{\Gamma \vdash n : \sigma} \\
\\
\frac
{\Gamma \vdash e : \text{word} \, s \quad \Gamma \vdash e' : \text{word} \, s }
{\Gamma \vdash e +^z e' : \text{word} \, s} \\
\\
\frac
{\Gamma \vdash e : \text{word} \, s \quad \Gamma \vdash e : \text{word} \, s}
{\Gamma \vdash e <^{sz} e' : \text{word} \, \text{char}} \\
\\
\frac
{}
{\Gamma \vdash \text{true} : \text{word} \, \text{char}} \\
\\
\frac
{}
{\Gamma \vdash \text{false} : \text{word} \, \text{char}} \\
\\
\frac
{\Gamma \vdash  e : \text{word} \, \text{char} \quad \Gamma \vdash e' : \sigma \quad \Gamma \vdash e'' : \sigma }
{\Gamma \vdash \text{if} (e) \{ e'\} \text{else} \{e''\} : \sigma} \\
\\
\frac
{\Gamma \vdash e : \text{void}* \quad \Gamma \vdash e' : \text{word} \, \text{size}}
{\Gamma \vdash \& e [e']^\sigma : \text{void}*} \\
\\
\frac
{\Gamma \vdash e : \sigma}
{\Gamma \vdash \text{continue} \, e : (\text{char}, \text{union} (\sigma, \tau))} \\
\\
\frac
{\Gamma \vdash e : \tau}
{\Gamma \vdash \text{continue} \, e : (\text{char}, \text{union} (\sigma, \tau))}
\end{gather*}

\end{document}
