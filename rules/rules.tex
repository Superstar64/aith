\documentclass {article}
\usepackage{mathtools}
\usepackage{mathabx}
\usepackage{hyperref}
\allowdisplaybreaks
\begin{document}

As of now, this document is mostly for myself. Its missing a lot of prose and explantions (and may eventually be discarded). \\

\tableofcontents

\section{Introduction}
All valid terms have a type who's type is of the form $ * $. \\
\\
All valid types have a type who's type is of the form of either $\boxvoid^\sigma$.
where $ \sigma $ is a $ \text{Unification} $. \\
\\
Variables in the type environment are annotated with their multiplicity. ie $ x :^1 ... $. \\
\\
Function literals are syntactically required to be top level only. Top level function types are internal. \\
\\
Overlines are used to mean 0 or more. \\
\\
After $ \beta $ reduction, terms not of type $ \sigma^\pi $ must either be a free variable or correspond to these $ \beta $ normal forms. \\
\\
\begin{gather*}
\begin{array}{c | c}
\text{Type} & \text{Term} \\
\hline
\sigma \xrightarrow{\uparrow}_\pi \tau & \lambda x. e \\
\forall \alpha : \kappa. \sigma & \Lambda \alpha. e  \\
\sigma \xrightarrow{\text{text}} \tau & \text{function}(x). e \\
\end{array}
\end{gather*} 

Some types use synatatic unificition, others use boolean unification (denoted by their kind).
Types which use boolean unification all permit the standard boolean operations(\texttt{true}, \texttt{false}, $\land$, $\lor$, etc).

\section{Object Categories}

\begin{align*}
e \tag{Terms} \\
x \tag{Variables} \\
\sigma, \tau, \pi, \kappa, \rho \tag{Types} \\
\alpha, \beta \tag{Type Variables} \\
\Gamma \tag{Type Environment} \\
\Phi \tag{Misc Environment} \\
\end{align*}

\subsubsection{Misc Categories}
\begin{align*}
n \tag{Numbers} \\
\text{sym} \tag{Symbol} \\
c \tag{Erasure} \\
c & \Coloneqq \\
& \circ \tag{Transparent} \\
& \bullet \tag{Concrete}
\end{align*}

\section{Judgment Forms}
The term rules are not syntax directed.
The type of a term must be a valid type ($ \Phi | \Gamma \vdash e : \sigma $, requires $ \Phi \vdash \sigma : \kappa $).
\begin{align*}
\Phi \mid \Gamma \vdash \, & e : \sigma \tag{\text{Term Validation}} \\
\Phi \vdash \, & \sigma : \kappa \tag{\text{Type Validation}} \\
\Phi \Vdash^c & \sigma \tag{Erasure Entailment}
\end{align*}

\section{Synonyms}

Synonyms for true / false in boolean unification.

\begin{gather*}
\begin{array}{c | c}
\text{Syntax} & \text{Synonym} \\
\hline
1 & \texttt{false} \\
\omega & \texttt{true} \\
\texttt{pure} & \texttt{false} \\
\end{array}
\end{gather*} 

\section{Typing Environments}
\begin{align*}
\Gamma \Coloneqq & \\
& \Gamma, x :^\pi \sigma \\
& \emptyset \\
\Phi \Coloneqq & \\
& \Phi, \alpha :^c \kappa \\
& \emptyset \\
\end{align*} \\
\\
Type environments can be multiplied by a multiplicity, following from Linear Haskell.
Except due to the representation of multiplicity as booleans, multiplication gets mapped to disjunction.
\begin{gather*}
\pi \Gamma = \{(x :^{\pi \lor  \pi'}) \, | \, (x :^{\pi'}) \in \Gamma \} \\
\Gamma, \Gamma' = (\Gamma \, \oplus \, \Gamma') \cup \{ (x^\omega) \, | \, (x :^\pi) \in \Gamma, (x :^{\pi'}) \in \Gamma' \}
\end{gather*}

\section{Meta}
Multiplicity uses boolean unification where $ \omega $ (\texttt{true}) is unrestricted and 1 (\texttt{false}) is linear.
\begin{align*}
\sigma, \tau, \pi, \kappa, \rho \Coloneqq & \\
& * \tag{Meta Type} \\
& \boxvoid^\sigma \tag{Kind} \\
& \top \tag{Top} \\
& \text{Syntactic} \\
& \text{Propositional} \\
& \text{Unification} \\
& \text{Multiplicity} \\
& \text{Label}
\end{align*}

\begin{gather*}
\frac
{}
{\Phi \vdash * : \boxvoid^\text{Syntactic}} \\
\\
\frac
{\Phi \vdash \sigma : \text{Unification}}
{\Phi \vdash \boxvoid^\sigma : \top} \\
\\
\frac
{}
{\Phi \vdash \text{Syntactic} : \text{Unification}} \\
\\
\frac
{}
{\Phi \vdash \text{Propositional} : \text{Unification}} \\
\\
\frac
{}
{\Phi \vdash \text{Unification} : \top} \\
\\
\frac
{}
{\Phi \vdash \text{Multiplicity} : \boxvoid^\text{Propositional}} \\
\\
\frac
{}
{\Phi \vdash \text{Label} : \top}
\end{gather*}

\subsection{Variables}
\begin{align*}
e \Coloneqq & \\
& x \tag{Variable} \\
& \dots \\
\sigma, \tau, \pi, \kappa, \rho \Coloneqq & \\
& \alpha \tag{Type Variable} \\
& \dots
\end{align*}
\begin{gather*}
\frac
{}
{\Phi \, | \, \omega (\Gamma \setminus \{x :^\pi \sigma\}), x :^\pi \sigma  \vdash x : \sigma}
\end{gather*}

\begin{gather*}
\frac
{(\alpha : \kappa) \in \Phi}
{\Phi \vdash \alpha : \kappa}
\end{gather*}

\subsection{Macro Linear Lambda Calculus}
\begin{align*}
e \Coloneqq & \\
& \lambda^\uparrow x. e \tag{Macro Lambda}\\
& e(^\uparrow e') \tag{Macro Application}\\
& \dots \\
\sigma, \tau, \pi, \kappa, \rho \Coloneqq & \\
& \sigma \xrightarrow{\uparrow}_\pi \tau \tag{Macro} \\
& \dots \\
\dots
\end{align*}

\begin{gather*}
\frac
{\Phi | \Gamma, x :^\pi \sigma \vdash e : \tau}
{\Phi | \Gamma \vdash \lambda^\uparrow x. e : \sigma \xrightarrow{\uparrow}_\pi \tau } \\
\\
\frac
{\Phi | \Gamma \vdash e_1 : \sigma \xrightarrow{\uparrow}_\pi \tau \quad \Phi | \Gamma' \vdash e_2 : \sigma}
{\Phi | \Gamma, \pi \Gamma' \vdash e_1(^\uparrow e_2) : \tau}
\end{gather*}

\begin{gather*}
\frac
{\Phi \vdash \sigma : * \quad \Phi \vdash \tau : * \quad \Phi \vdash \pi : \text{Multiplicity}}
{\Phi \vdash \sigma \xrightarrow{\uparrow}_\pi \tau  : * }
\end{gather*}

\subsection{System-F}
\begin{align*}
e \Coloneqq & \\
& \Lambda \alpha^. e \tag{Type Lambda} \\
& e (\sigma) \tag{Type Application} \\
& \dots \\
\sigma, \tau, \pi, \kappa, \rho \Coloneqq & \\
& \forall \alpha :^c \kappa. \sigma \tag{Type Poly}\\ 
& \dots
\end{align*}

\begin{gather*}
\frac
{\Phi, \alpha :^c \kappa \, | \, \Gamma \vdash e : \sigma}
{\Phi | \Gamma \vdash \Lambda \alpha. e : \forall \alpha :^c \kappa. \sigma } \\
\\
\frac
{\Phi | \Gamma \vdash e : \forall \alpha :^c \kappa. \sigma \quad \Phi \vdash \tau : \kappa \quad \Phi \Vdash^c \tau}
{\Phi | \Gamma \vdash e (\tau) : \sigma [\tau/\alpha] }
\end{gather*}

\begin{gather*}
\frac
{\Phi, \alpha :^\circ \kappa \vdash \sigma : * \quad \Phi \vdash \kappa : \boxvoid^\tau}
{\Phi \vdash \forall \alpha :^c \kappa. \sigma : * } \\
\end{gather*}

\section{Runtime}

\begin{align*}
\sigma, \tau, \pi, \kappa, \rho \Coloneqq & \\
& +^\rho_\pi \tag{Type}\\
& \text{Representation} \tag {Representation}\\
& \dots
\end{align*}

\begin{gather*}
\\
\frac
{\Phi \vdash \rho :^\tau \text{Representation} \quad \Phi \vdash \pi : \text{Multiplicity}}
{\Phi \vdash +^\rho_\pi : \boxvoid^\text{Syntactic}} \\
\\
\frac
{}
{\Phi \vdash \text{Representation} : \boxvoid^\text{Syntactic}}
\end{gather*}

\subsection{Regions}
todo add proper patterns to rules \\
Regions use boolean unification where \texttt{pure} (\texttt{false}) means using no regions.
\begin{align*}
e \Coloneqq & \\
& \text{letRGN} \, (\Lambda \alpha. e) \tag{Bind Region Type Variable (Unused)} \\
& \text{let}^\downarrow \, x = e; e' \tag{Runtime Let} \\
& \text{case} \, e \, \text{of} \{ x \to e'; x' \to' e''\} \tag{Case} \\
& \dots \\
\sigma, \tau, \pi, \kappa \Coloneqq & \\
& \sigma^\pi \tag{Region Effect} \\
& \text{IO} \tag{IO Region} \\
& \text{Region} \tag{Region} \\
& \dots
\end{align*}

\begin{gather*}
\frac
{\Phi , \alpha : \text{Region} \, | \, \Gamma \vdash e : \sigma^{\pi \lor (\alpha \land \rho)} \quad \alpha \notin \text{Free}(\sigma, \pi)}
{\Phi | \Gamma \vdash \text{letRGN} \, (\Lambda \alpha. e) : \sigma^\pi} \\
\\
\frac
{\Phi | \Gamma \vdash e : \tau^\pi \quad \Phi \vdash \tau : +^{\rho}_{\kappa} \quad
\Phi | \Gamma', x :^{\kappa} \tau^\texttt{pure} \vdash e' : \sigma^{\pi'} \quad
\Phi \Vdash^\bullet \rho}
{\Phi | \Gamma, \Gamma' \vdash \text{let}^\downarrow \, x = e; e' : \sigma^{\pi \lor \pi'}} \\
\\
\frac
{\Phi | \Gamma \vdash e : \tau^\pi \quad \Phi \vdash \tau : +^{\rho}_{\kappa} \quad
\overline{ \Phi | \Gamma', x :^{\kappa} \tau^\texttt{pure} \vdash e' : \sigma^\pi } \quad
\Phi \Vdash^\bullet \rho}
{\Phi | \Gamma, \overline {\Gamma'} \vdash \text{case} \, e \, \text{of} \{ \overline {x' \to  e' } \} : \sigma^{\pi \lor \overline{\pi'}}} \\
\end{gather*}

\begin{gather*}
\frac
{\Phi \vdash \pi : \text{Region} \quad \Phi \vdash \sigma : +^\rho_\tau}
{\Phi \vdash \sigma^\pi : *} \\
\\
\frac
{}
{\Phi \vdash \text{IO} : \text{Region}}
\end{gather*}

\begin{gather*}
\frac
{}
{\Phi \vdash \text{Region} : \boxvoid^\text{Propositional}} \\
\end{gather*}

\subsection{Boxed}
\begin{align*}
\sigma, \tau, \pi \Coloneqq & \\
& \text{unique} \, \sigma \\
& \sigma \, @ \, \pi \\
& - \tag{Boxed} \\
& \text{Pointer} \tag{Pointer Representation} \\
& \dots \\
\end{align*}

\begin{gather*}
\frac
{\Phi \vdash \sigma : -}
{\Phi \vdash \text{unique} \, \sigma : +^\text{Pointer}_1} \\
\\
\frac
{\Phi \vdash \sigma : - \quad \pi : \text{Region}}
{\Phi \vdash \sigma \, @ \, \pi : +^\text{Pointer}_\omega} 
\end{gather*}

\begin{gather*}
\frac
{}
{\Phi \vdash - : \boxvoid^\text{Syntactic}} \\
\\
\frac
{}
{\Phi \vdash \text{Pointer} : \text{Representation}} \\
\end{gather*}

\subsection{Pointers}
\begin{align*}
e \Coloneqq & \\
& *e \tag{Read Pointer} \\
& *e = e' \tag{Write Pointer} \\
& \& * \tag{Array to Pointer} \\
& \& e [e'] \tag{Array Increment} \\
& \dots \\
\sigma, \tau, \pi, \kappa, \rho \Coloneqq & \\
& \sigma* \tag{Pointer} \\
& \sigma[] \tag{Array Pointer} \\
\end{align*}

\begin{gather*}
\frac
{\Phi | \Gamma \vdash e : (\sigma* \, @ \pi')^\pi \quad \Phi \vdash \sigma : +^\rho_\omega \quad
\Phi \Vdash^\bullet \rho}
{\Phi | \Gamma \vdash * e  : \sigma^{\pi \lor \pi'}} \\
\\
\frac
{\Phi | \Gamma \vdash e : (\sigma* \, @ \pi')^\pi \quad \Phi | \Gamma \vdash e' : \sigma^{\pi''} \quad
\Phi \vdash \sigma : +^\rho_\omega \quad
\Phi \Vdash^\bullet \rho}
{\Phi | \Gamma \vdash *e  = e' : ()^{\pi \lor \pi' \lor \pi''} } \\
\\
\frac
{\Phi | \Gamma \vdash e : (\sigma[] @ \pi') ^ \pi}
{\Phi | \Gamma \vdash \& * e : (\sigma * @ \pi') ^ \pi}\\
\\
\frac
{\Phi | \Gamma \vdash e : (\sigma [] @ \pi')^\pi \quad 
\Phi |  \Gamma' \vdash e' : (\text{unsigned} \, \text{integer}(\text{native}))^\pi \quad
\Phi \vdash \sigma : +^\rho_\tau \quad \Phi \Vdash^\bullet \rho}
{\Phi | \Gamma, \Gamma' \vdash \&e [e'] : (\sigma [] @  \pi')^\pi}
\end{gather*}

\begin{gather*}
\frac
{\Phi \vdash \sigma : +^{\rho}_\sigma}
{\Phi \vdash \sigma* : -} \\
\\
\frac
{\Phi \vdash \sigma : +^{\rho}_\sigma}
{\Phi \vdash \sigma [] : -} \\
\end{gather*}

\subsection{Functions}
\texttt{function} and $ \sigma \xrightarrow{\text{text}}_\pi \tau $ are internal.
\begin{align*}
e \Coloneqq & \\
& \text{extern} \, \text{sym} \tag{Extern Function} \\
& e ^\downarrow (e') \tag{Function Pointer Application} \\
& \text{function} (x). e \tag{Function Literal} \\
& \dots \\
\sigma, \tau, \pi, \kappa, \rho \Coloneqq & \\
& \tau \xrightarrow{\downarrow}_\pi \sigma \tag{Function Pointer}\\
& \tau \xrightarrow{\text{text}}_\pi \sigma  \tag{Function Literal Type} \\
& \dots
\end{align*}

\begin{gather*}
\frac
{\Phi \vdash \tau : +^\rho_{\tau'} \quad \Phi \Vdash^\bullet \rho
\quad \Phi \vdash \sigma : +^{\rho'}_{\tau''} \quad \Phi \Vdash^\bullet \rho'}
{\Phi | \Gamma \vdash \text{extern} \, \text{sym} : (\tau \xrightarrow{\downarrow}_{\pi} \sigma)^\texttt{pure}} \\
\\
\frac
{\Phi | \Gamma \vdash e : (\sigma \xrightarrow{\downarrow}_{\pi'} \tau)^\pi \quad
\Phi | \Gamma' \vdash e' : \sigma^{\pi''} \quad
\Phi \vdash \tau : +^\rho_{\tau'} \quad \Phi \Vdash^\bullet \rho}
{\Phi | \Gamma, \Gamma' \vdash e^\downarrow (e') : \tau^{\pi \lor \pi' \lor \pi''}} \\
\\
\frac
{\Phi | \Gamma, x :^1 \sigma^\texttt{pure} \vdash e : \tau^\pi
\quad \Phi \vdash \sigma : +^\rho_{\tau'}
\quad \Phi \Vdash^\bullet \rho}
{\Phi | \Gamma \vdash \text{function}(x). e : \sigma \xrightarrow{\text{text}}_\pi \tau} \\
\end{gather*}

\begin{gather*}
\frac
{\Phi \vdash \sigma : +^{\rho}_{\kappa} \quad \Phi \vdash \tau : +^{\rho'}_{\kappa'} \quad \Phi \vdash \pi : \text{Region}}
{\Phi \vdash \sigma \xrightarrow{\downarrow}_\pi \tau : +^\text{Pointer}_\omega} \\
\\
\frac
{\Phi \vdash \sigma : +^{\rho}_{\kappa} \quad \Phi \vdash \tau : +^{\rho'}_{\kappa'} \quad \Phi \vdash \pi : \text{Region}}
{\Phi \vdash \sigma \xrightarrow{\text{text}}_\pi \tau : *} \\
\end{gather*}

\subsection{Tuples}
\begin{align*}
e \Coloneqq & \\
& (\overline e,) \tag{Tuple Introduction} \\
& \text{let}^\downarrow (\overline x) = e; e' \tag{Tuple Elimination} \\
& \dots \\
\sigma, \tau, \pi, \kappa, \rho \Coloneqq & \\
& (\overline \sigma,) \tag{Tuple} \\
& \text{Struct} \{ \overline \rho \} \tag{Struct Representation} \\
& \dots
\end{align*}

\begin{gather*}
\frac
{\Phi | \overline { \Gamma \vdash e : \sigma^\pi }}
{\Phi | \overline \Gamma  \vdash (\overline e,) : (\overline \sigma,)^{\lor \pi} } \\
\\
\frac
{\Phi | \Gamma \vdash e : (\overline \tau,)^\pi
\quad \Phi | \Gamma , \overline {x :^1 \tau^\texttt{pure}} \vdash e : \sigma^{\pi'} \quad
\overline {\Phi \vdash \tau : +^\rho_\kappa \quad \Phi \Vdash^\bullet \rho}}
{\Phi | \Gamma, \Gamma' \vdash \text{let}^\downarrow (\overline x) = e; e' : \sigma^{\pi \lor \pi'}} \\
\\
\frac
{\Phi \vdash \overline {\sigma : +^\kappa_\pi}}
{\Phi \vdash (\overline \sigma,) : +^{\text{Struct} \{\overline \kappa\}}_{\overline {\bigwedge \pi}}} \\
\\
\frac
{\Phi \vdash \overline {\rho :^\tau \text{Representation}}}
{\Phi \vdash \text{Struct} \{ \overline \rho \} :^{\bigwedge \tau} \text{Representation} } \\
\end{gather*}

\subsection{Choices}
\begin{align*}
\sigma, \tau, \pi, \kappa, \rho \Coloneqq & \\
& \text{Union} \{ \overline \rho \} \tag{Union Representation} \\
& \dots
\end{align*}

\begin{gather*}
\frac
{\Phi \vdash \overline {\rho :^\tau \text{Representation}}}
{\Phi \vdash \text{Union} \{ \overline \rho \} :^{\bigwedge \tau} \text{Representation} } \\
\end{gather*}

\subsection{Integer Arithmatic}
\begin{align*}
e \Coloneqq & \\
& n \tag{Numeric Literal} \\
& e + e' \tag{Addition} \\
& e - e' \tag{Subtraction} \\
& e * e' \tag{Multiplication} \\
& e / e' \tag{Division} \\
& e = e' \tag {Equality} \\
& e \neq e' \tag {Inequality} \\
& e < e' \tag {Less Then} \\
& e \leq e' \tag {Less Then Equal} \\
& e > e' \tag {Greater Then} \\
& e \geq e' \tag {Greater Then Equal} \\
& \dots \\
\sigma, \tau, \pi \Coloneqq & \\
& \rho \, \text{integer}(\rho) \tag{Number} \\
& \dots \\
\sigma, \tau, \pi, \kappa, \rho \Coloneqq & \\
& \text{Word} (\rho) \tag{Word Representation} \\
& 8 \tag{Byte Size} \\
& 16 \tag{Short Size} \\
& 32 \tag{Int Size} \\
& 64 \tag{Long Size} \\
& \text{native} \tag{Native Size} \\
& \text{signed} \tag{Signed} \\
& \text{unsigned} \tag{Unsigned} \\
& \text{Size} \tag{Size Sort}\\
& \text{Signedness} \tag{Signedness Sort}\\
& \dots \\
\end{align*}

\begin{gather*}
\frac
{\Phi \Vdash^\bullet \rho'}
{\Phi | \Gamma \vdash n : (\rho \, \text{integer}(\rho'))^\texttt{pure}} \\
\\
\frac
{\Phi | \Gamma \vdash e : (\rho \, \text{integer}(\rho'))^\pi \quad
\Phi | \Gamma, \Gamma' \vdash e' : (\rho \, \text{integer}(\rho'))^{\pi'} \quad
\Phi \Vdash^\bullet \rho \quad \Phi \Vdash^\bullet \rho'}
{\Phi | \Gamma, \Gamma' \vdash e + e' : (\rho \text{integer}(\rho'))^{\pi \lor \pi'}} \\
\dots \\
\\
\frac
{\Phi | \Gamma \vdash e : (\rho \, \text{integer}(\rho'))^\pi \quad
\Phi | \Gamma, \Gamma' \vdash e' : (\rho \, \text{integer}(\rho'))^{\pi'} \quad
\Phi \Vdash^\bullet \rho \quad \Phi \Vdash^\bullet \rho'}
{\Phi | \Gamma, \Gamma' \vdash e < e' : \text{Boolean}^{\pi \lor \pi'}} \\
\dots
\end{gather*}

\begin{gather*}
\frac
{\Phi \vdash \rho :^\tau \text{Signedness} \quad \Phi \vdash \rho' :^{\tau'} \text{Size}}
{\Phi \vdash \rho \, \text{integer}(\rho') : +^{\text{Word} (\rho')}_\omega } \\
\end{gather*}

\begin{gather*}
\frac
{\Phi \vdash \rho :^\tau \text{Size}}
{\Phi \vdash \text{Word} (\rho) :^\tau \text{Representation}} \\
\\
\frac
{}
{\Phi \vdash 8 : \text{Size}} \\
\\
\frac
{}
{\Phi \vdash 16 : \text{Size}} \\
\\
\frac
{}
{\Phi \vdash 32 : \text{Size}} \\
\\
\frac
{}
{\Phi \vdash 64 : \text{Size}} \\
\\
\frac
{}
{\Phi \vdash \text{native} : \text{Size}} \\
\\
\frac
{}
{\Phi \vdash \text{signed} : \text{Signedness}} \\
\\
\frac
{}
{\Phi \vdash \text{unsigned} : \text{Signedness}} \\
\\
\frac
{}
{\Phi \vdash \text{Size} : \boxvoid^\text{Syntactic}} \\
\\
\frac
{}
{\Phi \vdash \text{Signedness} : \boxvoid^\text{Syntactic}}
\end{gather*}

\subsection{Boolean Logic}
Note this has nothing to do with boolean unification.
\begin{align*}
e \Coloneqq & \\
& \text{true} \\
& \text{false} \\
& \text{if} (e) \{ e' \} \text{else} \{ e'' \} \\
\sigma, \tau, \pi, \kappa, \rho \Coloneqq & \\
& \text{Boolean}
\end{align*}

\begin{gather*}
\frac
{}
{ \Phi | \Gamma \vdash \text{true} : \text{Boolean}^\texttt{pure}} \\
\\
\frac
{}
{ \Phi | \Gamma \vdash \text{false} : \text{Boolean}^\texttt{pure}} \\
\\
\frac
{\Phi | \Gamma \vdash e : \text{Boolean}^\pi \quad \Phi | \Gamma' \vdash e' : \sigma^{\pi'} \quad \Phi | \Gamma' \vdash e'' : \sigma^{\pi''} }
{\Phi | \Gamma, \Gamma' \vdash \text{if} (e) \{ e' \} \text{else} \{ e'' \} : \sigma^{\pi \lor \pi' \lor \pi''} }
\end{gather*}

\begin{gather*}
\frac
{}
{ \Phi \vdash \text{Boolean} : +^{\text{Word}(8)}_\omega}
\end{gather*}

\subsection{Loops}
\begin{align*}
e \Coloneqq & \\
& \text{continue} \, e \tag{continue} \\
& \text{break} \, e \tag{break} \\
& \text{loop} (\text{let} \, x = e) \{ e' \} \tag{loop} \\
\sigma, \tau, \pi, \kappa, \rho \Coloneqq & \\
& \text{Step} \, \sigma \, \tau \tag{Loop Instruction}
\end{align*}

\begin{gather*}
\frac
{\Phi | \Gamma \vdash e : \sigma^\pi }
{\Phi | \Gamma \vdash \text{continue} \, e : (\text{Step} \, \tau \, \sigma)^\pi } \\
\\
\frac
{\Phi | \Gamma \vdash e : \tau^\pi \quad \Phi \vdash \sigma : +^\rho_\kappa \quad \Phi \Vdash^\bullet \rho}
{\Phi | \Gamma \vdash \text{break} \, e : (\text{Step} \, \tau \, \sigma)^\pi } \\
\\
\frac
{\Phi | \Gamma \vdash e : \sigma^\pi \quad
\Phi | \Gamma', x :^1 \sigma^\texttt{pure} \vdash e' : (\text{Step} \, \tau \, \sigma)^{\pi'}}
{\Phi | \Gamma, \omega \Gamma' \vdash \text{loop} (\text{let} \, x = e) \{ e' \} : \tau^{\pi \lor \pi'} } \\
\\
\frac
{\Phi \vdash \sigma : +^{\rho}_{\pi} \quad \Phi \vdash \tau : +^{\rho}_{\pi'} }
{\Phi \vdash \text{Step} \, \sigma \, \tau : +^{\text{Struct} (\text{Word} (8), \text{Union} (\rho, \rho') ) }_1 }
\end{gather*}
\section{Erasure Entailment}
\begin{gather*}
\frac
{}
{\Phi \Vdash^\circ \sigma} \\
\\
\frac
{(x :^\bullet \kappa) \in \Phi}
{\Phi \Vdash^\bullet x} \\
\\
\frac
{}
{\Phi \Vdash^\bullet \text{Pointer}} \\
\\
\frac
{\overline{\Phi \Vdash^\bullet \sigma}}
{\Phi \Vdash^\bullet \text{Struct} \{ \overline \sigma \}} \\
\\
\frac
{\overline{\Phi \Vdash^\bullet \sigma}}
{\Phi \Vdash^\bullet \text{Union} \{ \overline \sigma \}} \\
\\
\frac
{\Phi \Vdash^\bullet \sigma}
{\Phi \Vdash^\bullet \text{Word} (\sigma)} \\
\\
\frac
{}
{\Phi \Vdash^\bullet 8} \\
\\
\frac
{}
{\Phi \Vdash^\bullet 16} \\
\\
\frac
{}
{\Phi \Vdash^\bullet 32} \\
\\
\frac
{}
{\Phi \Vdash^\bullet 64} \\
\\
\frac
{}
{\Phi \Vdash^\bullet \text{native}} \\
\\
\frac
{}
{\Phi \Vdash^\bullet \text{unsigned}} \\
\\
\frac
{}
{\Phi \Vdash^\bullet \text{signed}} \\
\end{gather*}
\end{document}